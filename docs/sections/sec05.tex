\chapter{Conclusioni}

La realizzazione di questa ontologia dedicata ai prosumer ha dimostrato di essere un valido strumento di supporto nella gestione e analisi delle interazioni tra i consumatori e i produttori di energia elettrica. L'ontologia offre un insieme di classi, proprietà e regole SWRL che permettono di ottenere informazioni significative riguardanti i profili di carico, i sistemi di generazione e accumulo di energia, e i punti di misura dei prosumer.

Sebbene l'ontologia sia ancora parziale, la sua struttura e progettazione forniscono una solida base per un futuro sviluppo ed espansione. Per renderla ancora più efficace, sarebbe necessario arricchire il dominio con una maggiore varietà di prosumer, considerando anche tipologie di impianti di generazione e sistemi di accumulo diversi, al fine di ottenere una visione più completa e dettagliata delle interazioni tra i soggetti coinvolti nella rete elettrica.

L'aggiunta di nuove definizioni di classi, insieme alla creazione di proprietà e regole aggiuntive, consentirebbe di modellare scenari più complessi e specifici, migliorando la precisione delle inferenze effettuate dal reasoner. Inoltre, si potrebbe valutare l'integrazione di dati provenienti da fonti esterne, come misure effettive dei consumi e produzioni di energia, per rendere l'ontologia ancora più realistica e aderente al contesto reale.

Concludendo, riteniamo che l'obiettivo di sviluppare un'ontologia dedicata ai prosumer sia stato raggiunto con successo. L'ontologia offre un'infrastruttura solida e flessibile per la consultazione e l'analisi delle informazioni relative ai prosumer, fornendo così un supporto prezioso per la gestione ottimale della rete elettrica e promuovendo un utilizzo più efficiente e consapevole delle risorse energetiche disponibili.