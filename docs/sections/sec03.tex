\chapter{Modellazione e sviluppo dell'ontologia}

L'ontologia è stata creata partendo da una focalizzazione sulle entità primarie.

\section{Prosumer}
Il prosumer è l'entità principale dell'ontologia, ciò che andrà a contenere le entità caratterizzanti.


\begin{figure}[!ht]
    \centering
    \includegraphics[width=12cm]{images/individual_prosumer.png}
    \caption{Prosumer applicato a un individuo su Protègè.}
    \label{fig:individual_prosumer}
\end{figure}

Come si può notare nell'immagine \ref*{fig:individual_prosumer}, al prosumer vengono assegnati altri individui tramite le proprietà, grazie a ciò il reasoner è in grado di inferire la tipologia di configurazione.
In questo caso è di tipo Config01, siccome Energy Storage rispetta le caratteristiche della configurazione, come verrà mostrato nelle sezioni successive.

\subsection{Configuration}
Per stabilire la tipologia di configurazione, è stata creata la sottoclasse Configuration che a sua volta possiede le sottoclassi Config01, Config02 e Config03.

\begin{figure}[!ht]
    \centering
    \includegraphics[width=12cm]{images/pros_graph.png}
    \caption{Grafico creato su Protègè tramite il plugin OWLViz.}
    \label{fig:pros_graph}
\end{figure}

Per specificare le caratteristiche di ogni configurazione, sono state espresse le condizioni necessarie e sufficienti. Per ogni configurazione sono presenti:
\begin{verbatim}
    Prosumer 
     and (composedBy some Generator) 
     and (composedBy some Load) 
     and (composedBy some M1) 
     and (composedBy some M2) 
\end{verbatim}

Mentre, in aggiunta per ogni configurazione:
\begin{itemize}
    \item Configurazione 01: condizioni necessarie e sufficienti: \begin{verbatim}
        and (composedBy some 
            (StorageSystem 
             and (hasDirection some Monodirectional) 
             and (hasLocation some Production) 
             and (hasPowerType some DC)))
    \end{verbatim}
          condizioni necessarie: \begin{verbatim}
        Prosumer and (not (composedBy some M3))
    \end{verbatim}
    \item Configurazione 02: condizioni necessarie e sufficienti: \begin{verbatim}
        and (composedBy some 
            (StorageSystem 
             and (hasDirection some Bidirectional) 
             and (hasLocation some Production) 
             and (hasPowerType some (AC or DC))))
    \end{verbatim}
          condizioni necessarie: \begin{verbatim}
        Prosumer and (not (composedBy some M3))
    \end{verbatim}
    \item Configurazione 03: condizioni necessarie e sufficienti \begin{verbatim}
         and (composedBy some M3) 
         and (composedBy some 
            (StorageSystem 
             and (hasDirection some Bidirectional) 
             and (hasLocation some Post-Production)
             and (hasPowerType some AC)))
    \end{verbatim}
\end{itemize}


\section{Generator}
IL generatore rappresenta l'entità che produce energia ed è presente in tutti i prosumer.
L'energia prodotta dal generatore servirà a soddisfare il Load, a caricare lo Storage System e quella in eccesso verrà immessa nella rete.
\begin{figure}[H]
    \centering
    \includegraphics[width=12cm]{images/individual_generator.png}
    \caption{Istanza di un individuo di tipo generatore.}
    \label{fig:individual_generator}
\end{figure}

Per monitorare l'energia prodotta dal generatore, è stata definita un'istanza della classe "Time Series" (figura \ref{fig:individual_genmeas}), importata dall'ontologia ic-data. In questo modo si può definire un intervallo temporale e creare una sequenza di "Data Point"(anch'essa classe importata da ic-data), uno per ogni intervallo di tempo.
\begin{figure}[H]
    \centering
    \includegraphics[width=12cm]{images/individual_genmeas.png}
    \caption{Istanza di un individuo di tipo Time Series per il generatore.}
    \label{fig:individual_genmeas}
\end{figure}
Ogni singolo "GeneratorMeasurement" è quindi sottoclasse di Data Point e rappresenta un'osservazione dell'energia prodotta dal generatore in un intervallo di tempo specificato dal timestamp (figura \ref{fig:individual_genval3}).
\begin{figure}[H]
    \centering
    \includegraphics[width=12cm]{images/individual_genval3.png}
    \caption{Istanza di un individuo di tipo GeneratorMeasurement.}
    \label{fig:individual_genval3}
\end{figure}


\section{Load}
Il Load rappresenta il profilo di carico equivalente, anche detto consumo, del prosumer.
\begin{figure}[H]
    \centering
    \includegraphics[width=12cm]{images/individual_load.png}
    \caption{Istanza di un individuo di tipo Load.}
    \label{fig:individual_load}
\end{figure}
Analogamente per quanto fatto per il generatore, anche per il Load è stata definita un'istanza della classe "Time Series" (figura \ref{fig:individual_loadmes}) e ogni singolo "LoadMeasurement"
è sottoclasse di Data Point (figura \ref{fig:individual_loadval3}).
\begin{figure}[H]
    \centering
    \includegraphics[width=12cm]{images/individual_loadmes.png}
    \caption{Istanza di un individuo di tipo Time Series per il Load.}
    \label{fig:individual_loadmes}
\end{figure}
In questo modo si va a specificare l'energia consumata dal prosumer in un intervallo di tempo specificato dal timestamp, fornendo una visione dettagliata dell'utilizzo dell'energia da parte del prosumer nel corso del tempo.
\begin{figure}[H]
    \centering
    \includegraphics[width=12cm]{images/individual_loadval3.png}
    \caption{Istanza di un individuo di tipo LoadMeasurement.}
    \label{fig:individual_loadval3}
\end{figure}

\section{Storage System}
\begin{figure}[H]
    \centering
    \includegraphics[width=12cm]{images/individual_storagesystem.png}
    \caption{Istanza di un individuo di tipo StorageSystem.}
    \label{fig:individual_storagesystem}
\end{figure}
\begin{figure}[H]
    \centering
    \includegraphics[width=12cm]{images/individual_bat1.png}
    \caption{Istanza di un individuo di tipo Battery.}
    \label{fig:individual_bat1}
\end{figure}
\begin{figure}[H]
    \centering
    \includegraphics[width=12cm]{images/individual-batterysoc.png}
    \caption{Istanza di un individuo di tipo StateOfCharge.}
    \label{fig:individual-batterysoc}
\end{figure}

In figura \ref{fig:individual_batterym3} si possono notare le due Data Property inferite grazie alle regole SWRL descritte nella sezione \ref{sec:swrl}.
\begin{figure}[H]
    \centering
    \includegraphics[width=12cm]{images/individual-batterym3.png}
    \caption{Istanza di un individuo di tipo StateOfChargeMeasurement.}
    \label{fig:individual_batterym3}
\end{figure}

\section{Energy Meter}


