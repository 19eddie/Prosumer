\chapter{Introduzione}

L'ontologia del Prosumer è stata progettata specificamente per modellare ontologicamente gli aspetti rilevanti di un prosumer, un'entità caratterizzata dalla connessione a una rete elettrica e dotata di un profilo di consumo, un sistema di generazione e un sistema di accumulo dell'energia.

L'obiettivo principale dell'ontologia è quello di fornire una rappresentazione accurata e completa delle caratteristiche specifiche di un prosumer, concentrandosi esclusivamente sui suoi aspetti pertinenti. Ciò include la modellazione dei profili di consumo, dei sistemi di generazione e accumulo dell'energia, nonché la gestione dei contatori di scambio con la rete di distribuzione.

Inoltre, l'ontologia del Prosumer tiene conto delle diverse configurazioni possibili di un prosumer, comprese le varianti monodirezionali e bidirezionali dei sistemi di accumulo, distinguendo tra contatori M1, M2 e, quando presente, M3.

L'ontologia mira a fornire una base solida per l'interrogazione delle informazioni pertinenti riguardanti i prosumer, consentendo agli utenti di ottenere dettagliate informazioni sulla produzione, il consumo e lo stoccaggio dell'energia elettrica da parte di un prosumer.

Infine, è importante sottolineare che l'ontologia è stata costruita seguendo come riferimento e integrando, se necessario, altre ontologie, come SAREF (Smart Appliances REFerence) e InterConnect, al fine di garantire una maggiore interoperabilità e una rappresentazione più completa dei concetti legati al prosumer nell'ambito dell'energia elettrica.